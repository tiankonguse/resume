\documentclass[a4paper]{article}
\usepackage{CJK} %设置中文包
\usepackage{xcolor} % 用于配色
\usepackage{slashbox}
\usepackage{float}
\usepackage[ margin=2cm, headsep=.3cm]{geometry}
\usepackage{graphicx} %使用图形
\usepackage{amsmath} %使用数学库

\usepackage{amssymb}  
\usepackage{array}   
\usepackage{color}  
\usepackage{algorithm}  
\usepackage{booktabs}  
\usepackage{multirow}  
\usepackage{colortbl}   
\usepackage{array}
\usepackage{lipsum}% dummy text
\usepackage[colorlinks, linkcolor=black, anchorcolor=black, citecolor=black]{hyperref}

\usepackage{fancyhdr}
\pagestyle{fancy}
\fancypagestyle{plain}{
  \pagestyle{fancy}
}

%对齐方式
%左对齐 \begin{flushleft}...\end{flushleft}搜索
%居中 \begin{center}...\end{center}
%右对齐 \begin{flushright}...\end{flushright}

%打印公式符号
% 打印 % => \%
% 打印 ^ => \^{}
% 打印 _ => \_
%     + => \#
%     $ => \$
%     { => \{
%     } => \}
%     ~ => \~ ,推荐 $\sim$
%     & => \&
%     | => $|$
%     < => $<$
%     > => $>$
%     * => $*$
%


%页眉和页脚的左部,中部,右部
\lhead{}
\chead{}
\rhead{}
\lfoot{
$^{1}$最新简历信息请访问我的个人网站:\url{http://tiankonguse.com/resume/} \\
$^{2}$我的项目参见:\url{https://github.com/tiankonguse/} 
}
\cfoot{}
\rfoot{}

\definecolor{tabcolor}{rgb}{.105,.410,.113} 

\begin{document}

%使用CJK中文 ,仿宋格式
\begin{CJK*}{UTF8}{gbsn}



%全部左对齐
\begin{flushleft} 

%插入图片
%\begin{figure}
%	\centering{
%		\includegraphics[width=10cm]{photo3.eps}
%	}\\
%\end{figure}

\centering{\fontsize{30pt}{\baselineskip}
	\selectfont{
	%
		袁小康$^{1}$ $^{2}$\\
	%
	}
}

\begin{table}[h]
\begin{tabular}{p{4cm}p{4cm}p{4cm}p{4cm}}

\hline \multicolumn{2}{l}{出生日期:1990-5} & \multicolumn{2}{l}{籍贯:河南}\\


\hline \multicolumn{2}{l}{目前城市:长春市} & \multicolumn{2}{l}{毕业时间:2014 年毕业生}\\

\hline
E-mail:i@tiankonguse.com & 联系电话:13944097701 \\

\hline
学历:本科 & 毕业院校:东北师范大学 \\

\hline\multicolumn{2}{l}{求职期望 }\\

\hline
\multicolumn{2}{p{16cm}}{从事 WEB 技术开发方向或 c/c++技术方向,富有挑战性和创新性的工作}\\

\hline\multicolumn{2}{p{16cm}}{获得奖励 }\\
\hline 2013.6  & ACM-ICPC 东北四省赛(二等奖), 吉林省赛(一等奖) \\
\hline 2013.5  & ACM-ICPC 亚洲区 通化站邀请赛(银牌) \\
\hline 2012.11 & ACM-ICPC 亚洲区 杭州站(银牌) \\
\hline 2012.5  & ACM-ICPC 东北四省赛(二等奖), 吉林省赛(二等奖) \\
\hline 2011.4  & ACM-ICPC 亚洲区 北京站(铜牌) \\
\hline 2011.4  & 东北师范大学大学学生科研(特等奖) \\

\hline\multicolumn{2}{p{16cm}}{教育培训 }\\

\hline




\end{tabular}
\end{table}

%插入文本
\fontsize{10pt}{\baselineskip}\selectfont {

	



}

\end{flushleft}
\end{CJK*}
\end{document}
